\centerline{\Large\bfseries CryptMPI: A Fast Encrypted MPI Library}

\section{Scientific Background}
High performance computing (HPC) applications that process %highly-
sensitive data, such as medical, financial, and engineering documents have
to meet stringent security requirements. In order for such applications to execute in
the public cloud environment, it is imperative that the cloud infrastructure
supports both privacy and integrity. Many HPC applications utilize the Message
Passing Interface (MPI) library, the de facto library for message
passing applications. For MPI applications to run with security
guarantees in the public cloud environment, mechanisms must be
incorporated to support privacy and integrity in MPI communication.

Existing efforts to retrofit MPI libraries with encryption, however,
have introduced severe security flaws. For example,
ES-MPICH2~\cite{Ruan:2012:EMP:2197079.2197242}, the first such MPI library,
uses the weak ECB (Electronic Codebook) mode of operation that has known
vulnerabilities~\cite[page 89]{Book:KL14}.
In addition, no existing encrypted MPI libraries provide meaningful data
integrity, meaning that data could potentially be modified without being
detected. Consequently, it is urgent to revisit the problem by applying
the state-of-the-art theory and practice to properly
encrypt MPI communications.

In recent years, significant efforts have been put to improve the security,
usability, and performance of cryptographic libraries.
Popular cryptographic libraries, including OpenSSL~\cite{openssl},
BoringSSL~\cite{boringssl},
Libsodium~\cite{libsodium} and CryptoPP~\cite{cryptopp}, all received intensive
security review,
and some (OpenSSL and CryptoPP) even passed the Federal Information Processing
Standards  (FIPS) 140-2 validation. In addition, recent processors from
all major CPU vendors have introduced hardware support to speed up
the computationally intensive
cryptographic operations (e.g., Intel AES-NI instructions to accelerate the AES
algorithm, or the x86 CLMUL instruction set to improve the speed of finite-field
multiplications). All of these cryptographic libraries
now support hardware-accelerated cryptographic operations.

The advances in the networking infrastructure in data centers have
shifted the communication bottleneck from the network links to the network
end-points. As such, when incorporating security mechanisms in
the MPI library, the additional computation in the cryptographic operations
likely will introduce significant overheads to MPI communications,  %operations,
detrimental to the performance. To achieve high-performance and
secure MPI communication, it is thus critical to develop novel schemes to optimize
the MPI performance with encryption. 

In this research, we will develop CryptMPI, a high-performance secure MPI library.
CryptMPI uses AES GCM as well as the less expensive counter mode encryption technique
to support MPI communication with integrity and privacy.
CryptMPI will be built over MPICH (MPICH-3.2.1) and MVAPICH2 (MVAPICH2-2.3.2)
by incorporating the state-of-the-art encryption library BoringSSL and is intended
to be used by the broad community with MPI applications that require integrity
and privacy.

In our preliminary development, we have investigated various techniques to support
secure MPI communication. These include using IPSec, naively incorporating the
encryption into the MPI library \cite{Cluster:Naser19}, and developing more
advanced techniques such as pipelining and multithreading for secure
point-to-point communication and new collective algorithms for collective operations
with encryption. Like the development of other software libraries, implementation and
experimentation are essential for the new secure MPI library to be
practical and efficient. Xsede resources are critical in the performance
evaluation and tuning of CryptMPI and for ensuring that CryptMPI achieves high performance
in practice. Our ongoing work on this subject is supported by the NSF grant
CICI-1738912 and CRI-1822737.

\section{Research Objectives}

\section{Resource Usage Plan to achieve the research objectives}

\section{Justification of the allocation}

\section{Access to other CI resources}

